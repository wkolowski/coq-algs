% Opcje klasy 'iithesis' opisane sa w komentarzach w pliku klasy. Za ich pomoca
% ustawia sie przede wszystkim jezyk i rodzaj (lic/inz/mgr) pracy, oraz czy na
% drugiej stronie pracy ma byc skladany wzor oswiadczenia o autorskim wykonaniu.
\documentclass[declaration,mgr,english,shortabstract]{iithesis}

\usepackage[utf8]{inputenc}
\usepackage{graphicx}

%%%%% DANE DO STRONY TYTULOWEJ
% Niezaleznie od jezyka pracy wybranego w opcjach klasy, tytul i streszczenie
% pracy nalezy podac zarowno w jezyku polskim, jak i angielskim.
% Pamietaj o madrym (zgodnym z logicznym rozbiorem zdania oraz estetyka) recznym
% zlamaniu wierszy w temacie pracy, zwlaszcza tego w jezyku pracy. Uzyj do tego
% polecenia \fmlinebreak.
\englishtitle   {Formally verified algorithms and data structures in Coq: concepts and techniques}
\polishtitle    {Formalnie zweryfikowane algorytmy i struktury danych w Coqu: koncepty i techniki}
\polishabstract {Okasaki tylko lepiej}
\englishabstract{Okasaki but better}
% w pracach wielu autorow nazwiska mozna oddzielic poleceniem \and
\author         {Wojciech Kołowski}
% w przypadku kilku promotorow, lub koniecznosci podania ich afiliacji, linie
% w ponizszym poleceniu mozna zlamac poleceniem \fmlinebreak
\advisor        {narazienikt}
\date           {Czerwiec '20 chyba że koronawirus}                     % Data zlozenia pracy
% Dane do oswiadczenia o autorskim wykonaniu
%\transcriptnum {}                     % Numer indeksu
%\advisorgen    {dr. Jana Kowalskiego} % Nazwisko promotora w dopelniaczu
%%%%%

%%%%% WLASNE DODATKOWE PAKIETY
%
%\usepackage{graphicx,listings,amsmath,amssymb,amsthm,amsfonts,tikz}
%\usepackage{minted}
%
%%%%% WLASNE DEFINICJE I POLECENIA
%
%\theoremstyle{definition} \newtheorem{definition}{Definition}[chapter]
%\theoremstyle{remark} \newtheorem{remark}[definition]{Observation}
%\theoremstyle{plain} \newtheorem{theorem}[definition]{Theorem}
%\theoremstyle{plain} \newtheorem{lemma}[definition]{Lemma}
%\renewcommand \qedsymbol {\ensuremath{\square}}

%%%%%

\begin{document}

%%%%% POCZATEK ZASADNICZEGO TEKSTU PRACY

\chapter{A man, a plan, a canal - thesis} \label{ch1}

\section{Things to write about}
\begin{itemize}
    \item Introduction: differences between imperative and functional algorithms, differences between performance-oriented design and formal-correctness-oriented design.
    \item Literature review, Okasaki is old and bad for Coq, SF3 is shallow.
    \item Binary search trees: a case study to show the basic workflow and that it's not that obvious how to get basic stuff right.
    \item Design: we shouldn't require proofs in order to run programs. Ways of doing general recursion and, connected with it, functional induction as the way-to-go proof technique. Maybe something about the equations plugin. A word about classes, records and modules.
    \item Quicksort: in functional languages we have so powerful abstractions that we can actually implement \*algorithms\* and not just programs.
    \item  Braun mergesort: in order not to waste resources, we sometimes have to reify abstract patterns, like the splitting in mergesort.
    \item Cool data structures: ternary search trees, finger trees.
\end{itemize}

%%%%% BIBLIOGRAFIA

\begin{thebibliography}{1}
\end{thebibliography}

\end{document}