\documentclass{beamer}

\usepackage[utf8]{inputenc}
\usepackage{polski}
\usepackage[polish]{babel}
\usepackage{graphicx}

\usetheme{Darmstadt}
\useoutertheme{miniframes}
\makeatletter
  \beamer@compressfalse
\makeatother

\title{Formalnie zweryfikowane algorytmy w Coqu: koncepcje i techniki}
\author{Wojciech Kołowski}
\date{27 września 2021}

\begin{document}

\frame{\titlepage}

\addtocontents{toc}{\setcounter{tocdepth}{1}}
\frame{\tableofcontents}

\section{Historia}

\begin{frame}{Geneza pracy}
\begin{itemize}
	\item Około połowy 2017 roku postanowiłem sformalizować sobie kilka algorytmów i struktur danych. Tak powstał projekt żyjący pod adresem \url{https://github.com/wkolowski/coq-algs}
	\item W semestrze zimowym 2017/2018 brałem udział w kursie Algorytmy Funkcyjne i Trwałe Struktury Danych (czy jak on tam się zwał), który skłonił mnie do sięgnięcia po książkę Okasakiego i próbę sformalizowania tego, co w niej znajdę.
	\item Przez kolejne $+-$ dwa i pół roku projekt ewoluował i działy się w nim różne rzeczy, takie jak badania nad spamiętywaniem czy dowodzeniem przez reflekcję.
\end{itemize}
\end{frame}

\begin{frame}{Skąd odkrycia?}
\begin{itemize}
	\item Pewnego razu postanowiłem sformalizować pokaźny wachlarz przeróżnych algorytmów sortowania.
	\item Podjąłem się go częściowo z nudów, a częściowo żeby nieco systematyczniej zbadać, jak trudno (lub łatwo) formalizuje się algorytmy w Coqu i dowodzi ich poprawności.
	\item Wcześniejsze formalizacje algorytmów w tym projekcie nie dawały dobrej odpowiedzi na to pytanie, bo dotyczyły przypadkowo dobranych problemów, a rozwiązania pochodziły głównie z książki Okasakiego.
\end{itemize}
\end{frame}

\begin{frame}{Algorytmy sortowania motorem napędowym nauki}
\begin{itemize}
	\item Spora liczba algorytmów i jeszcze większa mnogość ich wariantów bardzo szybko wymusiły wysoce abstrakcyjne i modularne podejście, a potężny system typów Coqa umożliwił (niemal) bezproblemową realizację tej wizji.
	\item W miarę postępów formalizacji poznałem też całą masę uniwersalnych prawidłowości, które następnie przekułem w pracy na tytułowe koncepcje i techniki.
	\item Ostatecznie przekonałem się, że dla wprawnego użytkownika Coqa, uzbrojonego w moje koncepcje i techniki, dowodzenie poprawności algorytmów jest niewiele trudniejsze niż sama ich implementacja (choć jest dużo bardziej pracochłonne).
	\item Co więcej, zarówno implementacja jak i weryfikacja stają się łatwiejsze, gdy rozważamy je razem.
\end{itemize}
\end{frame}

\begin{frame}{Pareto wiecznie żywy}
\begin{itemize}
	\item Porównanie algorytmów sortowania okazało się więc być strzałem w dziesiątkę.
	\item Zadziałała też niezawodna zasada Pareto: mimo, że znaczna większość czasu i kodu poświęcona została podstawowym strukturom danych takim jak kolejki, sterty czy samobalansujące się drzewa wyszukiwań, to największym źródłem odkryć i najbardziej istotnym dla napisania pracy elementem projektu było właśnie porównanie algorytmów sortowania.
\end{itemize}
\end{frame}

\begin{frame}{Motywacje}
\begin{itemize}
	\item Poznawszy wspaniałe techniki radzenia sobie z formalizacją algorytmów stosowalne we wszystkich w zasadzie językach z typami zależnymi, postanowiłem podzielić się nimi ze światem.
	\item Głównym celem pracy było bardzo przyjazne i dostępne dla zwykłych śmiertelników opisanie tego wszystkiego, w formie tutorialowej zbliżonej stylem do najlepszych dydaktycznie znanych mi postów na blogach.
\end{itemize}
\end{frame}

\section{Streszczenie pracy}

\begin{frame}{Abstrakt}
\begin{center}
	Omawiamy sposoby specyfikowania, implementowania i weryfikowania funkcyjnych algorytmów, skupiając się raczej na dowodach formalnych niż na asymptotycznej złożoności czy faktycznej wydajności. Prezentujemy koncepcje i techniki, obie często opierające się na jednej kluczowej zasadzie -- reifikacji i reprezentacji, za pomocą potężnego systemu typów Coqa, czegoś, co w klasycznym, imperatywnym podejściu jest nieuchwytne, jak przepływ informacji w dowodzie czy kształt rekursji funkcji. Nasze podejście obszernie ilustrujemy na przykładzie quicksorta. Ostatecznie otrzymujemy solidną i ogólną metodę, którą można zastosować do dowolnego algorytmu funkcyjnego.
\end{center}
\end{frame}

\begin{frame}{Jak sformalizować funkcyjny algorytm?}
\begin{itemize}
	\item Praca prezentuje 15 technik przydatnych przy formalizacji algorytmów funkcyjnych.
	\item Liczba 15 jest trochę naciągana, zresztą nie o ilość chodzi.
	\item Najlepiej rozumieć te techniki jako podpunkty uniwersalnego planu, który prowadzi nas od problemu do formalnie zweryfikowanego algorytmu, który go rozwiązuje.
\end{itemize}
\end{frame}

\begin{frame}{Znajdź specyfikację problemu (sekcja 1.6 i rozdział 2)}
\begin{itemize}
	\item \#0: Dobra specyfikacja jest abstrakcyjna i prosta w użyciu.
	\item \#1: Dobra specyfikacja określa unikalny obiekt.
	\item \#2: \textit{Patchworkowa} specyfikacja może zostać ulepszona przy użyciu defunkcjonalizacji (czyli przy użyciu definicji induktywnych zamiast kwantyfikacji uniwersalnej, implikacji i funkcji).
	\item \#3: Czasem można znaleźć lepszą specyfikację skupiając się na innym aspekcie zagadnienia.
\end{itemize}
\end{frame}

\begin{frame}{Znajdź algorytm, który rozwiązuje problem}
\begin{itemize}
	\item Łatwiej powiedzieć niż zrobić.
	\item Praca nie porusza tematyki wymyślania czy projektowania algorytmów - zakładamy, że formalizujemy znany już algorytm.
\end{itemize}
\end{frame}

\begin{frame}{Zaimplementuj abstrakcyjny szablon algorytmu (rozdział 3)}
\begin{itemize}
	\item \#4: Przy implementacji pierwszego szablonu nie przejmuj się terminacją.
	\item \#6: Typy parametrów szablonu powinny zawierać wystarczająco \textit{evidence} (żeby np. nie popaść w ślepotę boolowską), ale nie za dużo.
	\item \#7: Jeżeli implementujesz wiele szablonów na raz (wielu powiązanych algorytmów), ich współdzielone komponenty powinny być parametrami szablonu. Pozostałe komponenty powinny być jego polami.
\end{itemize}
\end{frame}

\begin{frame}{Udowodnij terminację i poprawność (rozdziały 4 i 5)}
\begin{itemize}
	\item \#8: Użyj Metody Induktywnej Dziedziny (to mój rebranding metody Bove-Capretta) żeby zdefiniować lepszy szablon algorytmu, którego terminację łatwo udowodnić.
	\item \#12: W dowodzeniu niesamowicie przydatna jest technika \textit{proof by admission}, czyli formalny odpowiednik dowodu przez machanie rękami. W skrócie: jeżeli nie potrafisz czegoś udowodnić, załóż że to prawda. Później udowodnij lemat, który wypełni tę lukę, albo przyjmij dodatkowe założenie.
	\item \#9: Dowód terminacji sprowadza się do indukcji dobrze ufundowanej -- wystarczy znaleźć jakąś rozsądną relację dobrze ufundowaną lub miarę.
	\item \#11: Dowód poprawności szablonu sprowadza się do indukcji funkcyjnej.
\end{itemize}
\end{frame}

\begin{frame}{Dostarcz domyślną implementację (rozdziały 5 i 3)}
\begin{itemize}
	\item \#5: Dostarcz domyślną implementację.
	\item \#10: Upewnij się, że domyślna implementacja może zostać uruchomiona bez dowodzenia czegokolwiek (terminacji/poprawności) i że terminująca implementacja może zostać uruchomiona bez dowodzenia poprawności.
	\item \#13: Jeżeli twoja domyślna implementacja jest zbyt abstrakcyjna, dostarcz bardziej konkretną wersję.
	\item \#14: Do zdefiniowana domyślnej i konkretnej implementacji użyj type-driven development.
\end{itemize}
\end{frame}

\section{Podsumowanie i co dalej}

\begin{frame}{Główna kontrybucja pracy}
\begin{itemize}
	\item Główną kontrybucją pracy jest systematyzacja wielu technik, ich synteza w spójną metodę i zastosowanie tej metody konkretnie do problemu formalizacji algorytmów funkcyjnych.
	\item Uwaga: nie przypisuję sobie odkrycia w zasadzie żadnej z opisanych przeze mnie koncepcji i technik -- część była znana już wcześniej, część ma status pewnego folkloru, a część jest na tyle nieuchwytna, że trzeba je wymyślić na nowo, żeby porządnie je zrozumieć.
\end{itemize}
\end{frame}

\begin{frame}{Pomniejsze kontrybucje}
\begin{itemize}
	\item Udało mi się w przyjazny i zrozumiały sposób opisać także wiele znanych już technik, które dotychczas były nieopisane lub były opisane słabo.
	\item Indukcja funkcyjna, elegancka metoda dowodzenia właściwości funkcji rekurencyjnych, nie została nigdzie opisana w sposób przyjazny dla początkujących.
	\item Metoda Bove-Capretta (co za okropna nazwa!) została opisana w wielu pracach, ale nie są one przyjazne dla początkujących.
	\item \textit{Type-driven development} został przyjaźnie opisany, ale książka jest płatna i trochę przeterminowana.
	\item Spamowanie taktyką \texttt{admit} nie zostało nigdzie opisane jako systematyczna metoda dowodzenia i nie ma nawet oficjalnej nazwy (ja nazwałem to \textit{proof by admission}), a jego związki z \textit{Hole-driven development} są nie do końca oczywiste.
\end{itemize}
\end{frame}

\begin{frame}{Co dalej - ogólnie}
\begin{itemize}
	\item Zaprezentowane w pracy podejście do algorytmów funkcyjnych nie jest kompletne i można je rozszerzyć i poprawić.
	\item Nie znam żadnej ogólnej metody wymyślania specyfikacji.
	\item Pominięte zostały metody wymyślania/projektowania algorytmów, gdyż mieleniem ich zajmują się wszystkie możliwe kursy i książki, np. klasyczna książka Okasakiego i niedawno wydana \href{https://www.amazon.com/Algorithm-Design-Haskell-Richard-Bird/dp/1108491618}{\textit{Algorithm Design with Haskell}}.
	\item Analiza złożoności mogłaby także nieco zyskać na formalnym podejściu, choćby dlatego że możemy formalnie dowodzić zależności rekurencyjnych. Z drugiej strony analiza komplikuje się przez problemy związane np. z obecnością predykatu dziedzinowego.
\end{itemize}
\end{frame}

\begin{frame}{Co dalej - terminacja}
\begin{itemize}
	\item Wspomniałem o problemach standardowej Metody Induktywnej Dziedziny z funkcjami z zagnieżdżoną rekursją i rekursją wyższego rzędu, i jak sobie z tym poradzić, ale nie pokazałem, jak dokładnie należy to zrobić.
	\item Nie omówiłem żadnych metod konstruowania relacji dobrze ufundowanych.
	\item Nie omówiłem też alternatywnych metod dowodzenia terminacji jako HORPO (\textit{Higher Order Recursive Path Ordering}) czy \textit{Size-Change Principle}.
	\item Milczałem też o algorytmach korekurencyjnych i dowodzeniu ich produktywności, choć to temat raczej badawczy niż dydaktyczno-folklorystyczny.
\end{itemize}
\end{frame}

\begin{frame}{Co dalej - bujanie w chmurach}
\begin{itemize}
	\item Zdaje się, że istnieje linia badań nad ``obliczaniem'' algorytmu ze specyfikacji za pomocą rozumowań równaniowych. W połączeniu z techniką wymyślania specyfikacji byłoby to naprawdę potężne combo.
	\item Ciekawą rzeczą jest Kombinatoryka Analityczna, która pozwala obliczyć liczbę struktur (np. drzew) danego rodzaju z samej definicji typu induktywnego, dzięki czemu stanowi pomost między specyfikacją a analizą złożoności.
	\item Testowanie może się wydawać zbędne, jednak \textit{property-based testing} jest bardzo dobry w znajdowaniu kontrprzykładów, co może znacznie skrócić czas od popełnienia błędu do wykrycia go -- moglibyśmy zrobić to już na etapie domyślnej implementacji, czyli niedługo po zaimplementowaniu pierwszego szablonu, zamiast dopiero na końcu, przy dowodzie poprawności.
\end{itemize}
\end{frame}

\end{document}